\documentclass[11pt,a4paper]{article}
\usepackage[margin=2.5cm]{geometry}

\usepackage{amsmath, amssymb, amsfonts, amsthm}
\usepackage{graphicx}
\usepackage{hyperref}
\usepackage{tikz}
\usepackage{tikz-qtree}


% new theorem like environments
\newtheorem{proposition}{Proposition}
\newtheorem{lemma}{Lemma}
\newtheorem{theorem}{Theorem}
\newtheorem{corrolary}{Corrolary}

% macro for defining new complexity classes
\newcommand{\classX}[1]{\ensuremath{\text{\textsf{\textbf{#1}}}}} 
\newcommand{\classP}{\classX{P}}
\newcommand{\classNP}{\classX{NP}}
\newcommand{\NPC}{\classX{NP-complete}}
\newcommand{\coNP}{\classX{coNP}}
\newcommand{\EXP}{\classX{EXP}}
\newcommand{\coEXP}{\classX{coEXP}}
\newcommand{\PSPACE}{\classX{PSPACE}}
\newcommand{\NPH}{\classX{NP-hard}}


\title{Time Hierarchy theorem for quantum turing machines}
\author{Syed Mujtaba Hassan}

\begin{document}
\maketitle
% \begin{abstract}
% \end{abstract}

\section*{Literature Review}
Turing machines were first introduced by Alan Turing in 1936 to solve the Entscheidungsproblem \cite{11}.
It was later shown to be equivalent to every other physically realizable model of computation \cite{13}\cite{14}\cite{10}, and thus used as the standard model of computation.
The idea of a quantum turing machine was first introduced by Benioff in 1980 \cite{15}\cite{16}. 
The idea was later formalized by Deutsch in 1985 \cite{9}. Later Bernstein and Vazirani introduced their own formalism for quantum turing machine which is generally known as B\&V-QTM \cite{5}\cite{4}\cite{8}.
This introduced the field of quantum complexity theory where we study complexity classes for quantum turing machines \cite{8}\cite{6}\cite{5}.
For deterministic turing machines we have Time Hierarchy theorems which shows the separation between different time classes \cite{6}\cite{10}.
For deterministic turing machine the Time Hierarchy theorem states that $\text{DTIME}(o(\frac{f(n)}{\log(f(n))})) \subset \text{DTIME}(O(f(n)))$, for a time constructible function $f$ \cite{6}\cite{10}. 
Similiar theorems also exits for other syntacital models such as nondeterminitic turing machines \cite{6}\cite{10}. 
No time hierarchy theorem exits for quantum turing machines. 
It is to note that no general time hierarchy theorem exits for any symantic model such as probabilistic turing machines or quantum turing machines, however theorems exits for such model with some advice bits \cite{3}.
One main challenge in devising a time hierarchy theorem for quantum turing machines is to decide what formalism to use, however Guerrini Stefano, Martini Simone and Masini Andrea devise their formalism which aims to fill the gap in Deutsch's and Bernstein and Vazirani's formalism for quantum turing machines \cite{4}.
Another major issue is that non of the formalisms defines a universal quantum turing machine in the classical sense \cite{2}. 
Furthermore as quantum turing machines are probabilistic in nature the output after measurement might vary each time the machine is ran on the same input \cite{8}. Due to these two reasons daignalization argument is hard to construct for quantum turing machines.
In our project we will use the superposed configurations and the time evolution operator for a quantum turing machine to construct a daignalization argument.
\bibliographystyle{plain} 
\bibliography{refs} 
\nocite{*}% Show all bib entries - both cited and uncited; comment this line to view only cited bib entries;


\end{document}
